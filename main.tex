\documentclass[a4paper]{article}

%%%%%%%%%%%%%%%%%%page size%%%%%%%%%%%%%%%%%%
% \paperwidth=65cm
% \paperheight=160cm

%%%%%%%%%%%%%%%%%%%Package%%%%%%%%%%%%%%%%%%%
\usepackage[margin=3cm]{geometry}
\usepackage{mathtools,amsthm,amssymb}
\usepackage{yhmath}
\usepackage{graphicx}
\usepackage{fontspec}
\usepackage{titlesec}
\usepackage{titling}
\usepackage{fancyhdr}
%\usepackage{tabu}
%\usepackage{booktabs}
\usepackage[square, comma, numbers, super, sort&compress]{natbib}
\usepackage[unicode, pdfborder={0 0 0}, bookmarksdepth=-1]{hyperref}
\usepackage[usenames, dvipsnames]{color}
\usepackage[shortlabels, inline]{enumitem}
\usepackage{xpatch}

\usepackage{float}
\usepackage{caption}
\usepackage{subcaption}

\usepackage{optidef}

%\usepackage[abbreviations]{siunitx}
\usepackage{mdframed}
%\usepackage{multicol}
%\usepackage{dsfont}
%\usepackage{setspace}
%\usepackage{tabto}
%\usepackage{soul}
%\usepackage{ulem}
%\usepackage{wrapfig}
%\usepackage{floatflt}

%%%%%%%%%%%%%%%%%%%TikZ%%%%%%%%%%%%%%%%%%%%%%
\usepackage{tikz}
\usepackage{circuitikz}
\usetikzlibrary{calc}
\usetikzlibrary{arrows,shapes}
\usetikzlibrary{positioning}
\tikzstyle{every picture}+=[remember picture]

% chinese environment with xelatex
\usepackage[CheckSingle, CJKmath]{xeCJK}
\usepackage{CJKulem}
\setCJKmainfont[BoldFont=Noto Sans CJK TC Bold]{Noto Serif CJK TC}
\setCJKsansfont[BoldFont=Noto Sans CJK TC Bold]{Noto Serif CJK TC}
\setCJKmonofont[BoldFont=Noto Sans CJK TC Bold]{Noto Serif CJK TC}

%%%%%%%%%%%%%%%%%font size%%%%%%%%%%%%%%%%%%%
%\def\normalsize{\fontsize{10}{15}\selectfont}
%\def\large{\fontsize{12}{18}\selectfont}
%\def\Large{\fontsize{14}{21}\selectfont}
%\def\LARGE{\fontsize{16}{24}\selectfont}
%\def\huge{\fontsize{18}{27}\selectfont}
%\def\Huge{\fontsize{20}{30}\selectfont}

%%%%%%%%%%%%%%%Theme Input%%%%%%%%%%%%%%%%%%%
%\input{themes/chapter/neat}
%\input{themes/env/problist}

%%%%%%%%%%%titlesec settings%%%%%%%%%%%%%%%%%
%\titleformat{\chapter}{\bf\Huge}
            %{\arabic{section}}{0em}{}
%\titleformat{\section}{\centering\Large}
            %{\arabic{section}}{0em}{}
%\titleformat{\subsection}{\large}
            %{\arabic{subsection}}{0em}{}
%\titleformat{\subsubsection}{\bf\normalsize}
            %{\arabic{subsubsection}}{0em}{}
%\titleformat{command}[shape]{format}{label}
            %{gutter}{before}[after]

%%%%%%%%%%%%variable settings%%%%%%%%%%%%%%%%
%\numberwithin{equation}{section}
%\setcounter{secnumdepth}{4}
%\setcounter{tocdepth}{1}
%\setcounter{section}{0}
%\graphicspath{{images/}}

%%%%%%%%%%%%%%%page settings%%%%%%%%%%%%%%%%%
%\newcolumntype{C}[1]{>{\centering\arraybackslash}p{#1}}
\setlength{\headheight}{15pt}  % with titling
\setlength{\droptitle}{-1.5cm}
%\posttitle{\par\end{center}}  % distance between title and content
\parindent=0pt % indent size
%\parskip=1ex    % line space
%\pagestyle{empty}  % empty: no page number
%\pagestyle{fancy}  % fancy: fancyhdr

% use with fancygdr
%\lhead{\leftmark}
%\chead{}
%\rhead{}
%\lfoot{}
%\cfoot{}
%\rfoot{\thepage}
%\renewcommand{\headrulewidth}{0.4pt}
%\renewcommand{\footrulewidth}{0.4pt}

%\fancypagestyle{firststyle}
%{
  %\fancyhf{}
  %\fancyfoot[C]{\footnotesize Page \thepage\ of \pageref{LastPage}}
  %\renewcommand{\headrule}{\rule{\textwidth}{\headrulewidth}}
%}

%%%%%%%%%%%%%%%renew command%%%%%%%%%%%%%%%%%
% \renewcommand{\contentsname}{Table of Content}
% \renewcommand{\refname}{Reference}
\renewcommand{\abstractname}{\LARGE Abstract}

%%%%%%%%symbol and function settings%%%%%%%%%
\DeclarePairedDelimiter{\abs}{\lvert}{\rvert}
\DeclarePairedDelimiter{\norm}{\lVert}{\rVert}
\DeclarePairedDelimiter{\inpd}{\langle}{\rangle} % inner product
\DeclarePairedDelimiter{\ceil}{\lceil}{\rceil}
\DeclarePairedDelimiter{\floor}{\lfloor}{\rfloor}
\DeclareMathOperator{\adj}{adj}
\DeclareMathOperator{\sech}{sech}
\DeclareMathOperator{\csch}{csch}
\DeclareMathOperator{\arcsec}{arcsec}
\DeclareMathOperator{\arccot}{arccot}
\DeclareMathOperator{\arccsc}{arccsc}
\DeclareMathOperator{\arccosh}{arccosh}
\DeclareMathOperator{\arcsinh}{arcsinh}
\DeclareMathOperator{\arctanh}{arctanh}
\DeclareMathOperator{\arcsech}{arcsech}
\DeclareMathOperator{\arccsch}{arccsch}
\DeclareMathOperator{\arccoth}{arccoth}

\DeclareMathOperator*{\argmin}{arg\,min}

\newcommand*{\Nb}{\mathbb{N}}
\newcommand*{\Zb}{\mathbb{Z}}
\newcommand*{\Qb}{\mathbb{Q}}
\newcommand*{\Rb}{\mathbb{R}}
\newcommand*{\Cb}{\mathbb{C}}
\newcommand*{\Xc}{\mathcal{X}}

\DeclareMathOperator{\Pb}{\mathbb{P}}
\DeclareMathOperator{\Eb}{\mathbb{E}}
\DeclareMathOperator{\Ord}{\mathcal{O}}

%\newcommand*{\KL}[2]{D_\text{KL}\left( {#1} \middle\| {#2} \right)}
\DeclarePairedDelimiterXPP{\KL}[2]{D_\text{KL}}\lparen\rparen{}{{#1} \delimsize\Vert {#2}}
\DeclarePairedDelimiterXPP{\Prob}[1]{\Pb}\lbrace\rbrace{}{#1}
\DeclarePairedDelimiterXPP{\Ev}[1]{\Eb}\lbrack\rbrack{}{#1}
\DeclarePairedDelimiterXPP{\Evr}[2]{\Eb_{#1}}\lbrack\rbrack{}{#2}
%\newcommand*{\Evr}[2]{\Eb_{#1} \left[{#2}\right]}

% just to make sure it exists
\providecommand\given{}
% can be useful to refer to this outside \Set
\newcommand*\SetSymbol[1][]{%
  \nonscript\:#1\vert
  \allowbreak
  \nonscript\:
\mathopen{}}
\DeclarePairedDelimiterX\Set[1]\{\}{%
  \renewcommand\given{\SetSymbol[\delimsize]}
  \,#1\,
}

\DeclarePairedDelimiterX\Gen[1]{\langle}{\rangle}{%
  \renewcommand\given{\SetSymbol[\delimsize]}
  \,#1\,
}

%%%%%%%%%%%%%%%%%%%%%%%%%%%%%%%%%%%%%%%%%%%%
%\renewcommand{\proofname}{\bf proof:}
\newtheoremstyle{mystyle}% custom style
  {6pt}{15pt}%      top and bottom margin
  {}%               content style
  {}%               indent
  {\bf}%            head style
  {.}%              after head
  {1em}%            distance between head and content
  {}%               Theorem head spec (can be left empty, meaning 'normal')

\theoremstyle{mystyle}
\newtheorem{theorem}{Theorem}
\newtheorem{coro}{Corollary}
\newtheorem{definition}{Definition}
\newtheorem{lemma}{Lemma}
\newtheorem*{lemma*}{Lemma}
\newtheorem{property}{Property}

%%%%%%%%%%%%%%Title information%%%%%%%%%%%%%%
\title{Advanced Algorithms week 12} 
\author{Yao-Wen Mao}
\date{\today}

\begin{document}
\maketitle
% \thispagestyle{empty}
% \thispagestyle{fancy}
% \tableofcontents
%%%%%%%%%%%%%include file here%%%%%%%%%%%%%%%
\section*{Notation}
\begin{itemize}
  \item $\Prob{ \text{event} }$ denote the probability of the event.
  \item $\Ev{X}$ denote the expected value of the random variable $X$.
  \item Let $P, Q$ be two distribution defined on the sample space $\Xc$.
    $\KL{P}{Q}$ denote the KL-divergence from $Q$ to $P$.
    \[
      \KL{P}{Q} = \Evr*{X \sim P(\cdot)}{\log \frac{P(x)}{Q(x)}}
    \]
  \item Use $B_p$ denote the Bernoulli distribution:
    $ B_p(1) = p, B_p(0) = 1 - p. $
  \item We write $\KL{p}{q} = \KL{B_p}{B_q}$ for convenience.
\end{itemize}

\setcounter{section}{-1}
\section{Recap - Hashing}
\begin{itemize}
  \item {\tt INS}/{\tt DEL}/{\tt search} in $\Ord(1)$ time.
  \item Uniform hashing, put $n$ items into $n$ slots.
    $\max n_i = \Theta\left( \frac{\log n}{\log \log n} \right)$,
    where $n_i$ is \# of items in the $i$-th slot.
  \item 2-universal Hashing.
  \item Perfect Hashing.
\end{itemize}

\section{Chernoff Bound}
\begin{theorem}[Chernoff Bound]
  Let $X_1, \dots, X_n$ be $n$ independent binary random variables with
  $\Ev{X_i} = p_i$, i.e.
  \[
      \Prob{X_i = 1} = p_i \qquad \Prob{X_i = 0} = 1 - p_i
  \]
  Let $X = \sum_{i=1}^n X_i, \mu = \Ev{X} = \sum_{i=1}^n p_i$ and $p = \frac{\mu}{n}$.
  Set $\lambda = n\delta$, then
  \begin{align*}
    \Prob{ X \ge \mu + \lambda } \le \exp(-n \KL{p+\delta}{p}), & \quad
    0 \le \delta \le 1 - p \\
    \Prob{ X \le \mu - \lambda } \le \exp(-n \KL{p-\delta}{p}), & \quad
    0 \le \delta \le p
  \end{align*}

  \begin{proof}
    First, we know the Markov's inequality: $\Prob{X \ge a} \le \frac{\Ev{X}}{a}$.
    Then consider the random variable $e^{tX}$ for some $t > 0$, we have
    \[
      \Prob{X \ge a} = \Prob{e^{tX} \ge e^{ta}} \le \frac{\Ev*{e^{tX}}}{e^{ta}}
    \]
    Let $q \in [0, 1]$. We can write
    \begin{align*}
      \Prob{X \ge nq} &\le e^{-tnq} \Ev*{e^{tX}} \\
                     &= e^{-tnq} \left( \prod_{i=1}^n \Ev*{e^{tX_i}} \right)
                     && \text{(Since $X_i$ are independent)}\\
                     &\le e^{-tnq} \left(\frac{1}{n} \sum_{i=1}^n \Ev*{e^{tX_i}} \right)^n
                     && \text{(GM-AM inequality)}\\
                     &\le e^{-tnq} \left(\frac{1}{n} \sum_{i=1}^n p_i e^t + (1 - p_i) \right)^n \\
                     &\le e^{-tnq} (p e^t + (1 - p))^n \\
                     &= \big(p e^{(1-q)t} + (1-p)e^{-qt}\big)^n
    \end{align*}
    So $\Prob{X \ge nq} \le \inf\limits_{t > 0} \big(p e^{(1-q)t} + (1-p)e^{-qt}\big)^n$.
    Let $f(t) = p e^{(1-q)t} + (1-p)e^{-qt}$, then we can compute
    \[
      \frac{\mathrm{d} f}{\mathrm{d} t}
      = (1-q)p e^{(1-q)t} - q(1-p)e^{-qt} = 0
      \implies 
      e^t = \frac{q}{1-q}\frac{1-p}{p}
    \]
    then we can get
    \begin{align*}
      \Prob{X \ge nq}
      &\le \left(\left(\frac{q}{1-q}\frac{1-p}{p} \right)^{-q} \frac{1-p}{1-q}\right)^n \\
      &\le \exp \left( -n \left( q \ln \frac{q}{1-q}\frac{1-p}{p} + \ln \frac{1-q}{1-p} \right)\right)\\
      &= \exp \left( -n \left( q \ln \frac{q}{p} + (1-q) \ln \frac{1-q}{1-p} \right)\right)\\
      &= \exp(-n \KL{q}{p}).
    \end{align*}
    Set $q = p + \delta \leadsto nq = \mu + \lambda$, then
    \[
      \Prob{X \ge \mu + \lambda} \le \exp(-n \KL{p+\delta}{p}).  \qedhere
    \]
  \end{proof}
\end{theorem}

\begin{coro} \mbox{}
  \begin{itemize}
    \item $\Prob{X \ge \mu + \lambda} \le \exp\left(\frac{-2\lambda^2}{n}\right)$.
    \item $\Prob{X \ge \mu - \lambda} \le \exp\left(\frac{-2\lambda^2}{n}\right)$.
    \item $\Prob{X \ge (1+\beta)\mu} \le \exp\left(\frac{-\mu\beta^2}{2+\beta}\right),
      \quad \beta > 0$.
    \item $\Prob{X \ge (1+\beta)\mu} \le \exp\left(\frac{-\mu\beta^2}{3}\right),
      \quad 0 < \beta \le 1$.
    \item $\Prob{X \ge (1-\beta)\mu} \le \exp\left(\frac{-\mu\beta^2}{2}\right),
      \quad 0 < \beta < 1$.
    \item $\Prob{X \ge e\mu} \le e^{-\mu}$.
  \end{itemize}
\end{coro}

\begin{mdframed}
  Note that $\Prob{X \ge e\mu} \le e^{-\mu}$ is deduced from
  \[
    \Prob{X \ge (1+\beta)\mu} \le \left(\frac{e^\beta}{(1+\beta)^{(1+\beta)}}\right)^\mu
  \]
\end{mdframed}

\newpage

\section{Power of 2 Choices}
We want to put $n$ balls into $n$ bins.
\begin{itemize}
  \item For any ball $j$, pick $2$ bins $a_{j1}, a_{j2}$ uniformly at random.
  \item For $j = 1$ to $n$:
    \begin{itemize}
      \item Let $b_{ji}$ denote the number of balls in the $i$-th bin
        at the moment.
      \item put ball $j$ into the bin $\argmin\limits_{s \in \{ a_{i1}, a_{i2} \}} b_{js}$,
        i.e. the bin with less balls at the moment.
    \end{itemize}
\end{itemize}
Let $S_i$ denote the set of balls in the $i$-th bin and $b_i = \abs{S_i}$.
Then ${\color{blue} \Ev{ \max_i b_i } = \Ord(\log \log n)}$.

\begin{proof} \mbox{}
  \begin{itemize}
    \item Let $B_k \triangleq \abs[\big ]{\Set{ i \given b_i \ge k }}$,
      i.e. the number of bins with balls $\ge k$.
    \item Now we prove by induction on $k$:
      \begin{enumerate}
        \item Assume $B_k \le \beta_k$ with high probability
          ($\Prob{B_k > \beta_k} \le \frac{1}{n^2}$).
        \item For any ball $j$, assume $j \in S_t$. We say it is red if
          $\abs{\Set{ i \given i < j \text{ and } i \in S_t}} \ge k$,
          i.e. there are at least $k$ balls put into $t$-th bin before $j$.
        \item $\Prob{ \text{ball $j$ is red }} \le \left(\frac{\beta_k}{n}\right)^2$
          since we need to pick two bins with at least $k$ balls.
        \item $\Ev{B_{k+1}} \le \Ev{ \text{\# of red balls} } \le
          n\left(\frac{\beta_k}{n}\right)^2$.
        \item Using chernoff bound, we get
          $\Prob{B_{k+1} \ge e\frac{\beta_k^2}{n}} \le e^{-\frac{\beta_k^2}{n}}$.
          If $e^{-\frac{\beta_k^2}{n}}$ is small enough, then we can set
          $\beta_{k+1} = e\frac{\beta_k^2}{n}$.
        \item Then $\frac{\beta_{k+1}}{n} = e\left(\frac{\beta_k}{n}\right)^2
          \implies \frac{\beta_{k+m}}{n} = e^{-1} \left(\frac{e\beta_k}{n}\right)^{2^m}$.
          Notice that $\beta_6 \le \frac{n}{6} \le \frac{n}{2e}
          \implies \frac{\beta_{6+m}}{n} \le e^{-1} \frac{1}{2^{2^m}}$.
          So when $m = \Ord(\log \log n)$, we can find $\beta_{6+m} < 1$.
        \item If the prob. in 5 is not small enough, say
          $e^{-\frac{\beta_k^2}{n}} > \frac{1}{n^2} \implies \beta_k < \sqrt{2n \ln n}$.
        \item Then $\Ev{B_{k+1}} \le \frac{\beta_k^2}{n} < 2 \ln n
          \implies \Prob{B_{k+1} \ge 6 \ln n} < \frac{1}{n}$.
        \item $\Prob{B_{k+2} \ge 1} \le n \left(\frac{6 \ln n}{n}\right)^2
          = \Ord\left(\frac{\log^2 n}{n}\right)$.
      \end{enumerate}
    \item By 6 and 9, we can conclude that 
      ${\color{blue} \Ev{ \max_i b_i } = \Ord(\log \log n)}$.
      \qedhere
  \end{itemize}
\end{proof}

\section{Dynamic Resizing}
想法:偶爾把整個 hash table 打掉重建,並讓複雜度均攤到各個操作上。

{\tt INS}/{\tt DEL}/{\tt search}: the same.

Let $n$ denote the number of elements, $m$ denote the number of slots.

\begin{enumerate}
  \item When $n = m$, reconstruct the hash table with size $2m$.
  \item When $n \le \frac{m}{4}$, reconstruct the hash table with size $\frac{m}{2}$.
\end{enumerate}

\underline{Observation:} After each reconstruction, $n = \frac{m}{2}$.

Now we check the amortized complexity of two types of reconstruction:
\begin{enumerate}
  \item If $n = m$, then there are at least $\frac{m}{2}$ operations between two
    reconstruction. The cost of reconstruction is $\Ord(m)$, so the amortized
    cost is $\Ord(1)$.
  \item If $n \le \frac{m}{4}$, then there are at least $\frac{m}{4}$ operations
    between two reconstruction. The cost of reconstruction is $\Ord(m)$, so the
    amortized cost is $\Ord(1)$.
\end{enumerate}

\section{Consistent Hashing}
\begin{enumerate}
  \item Pick $a_1, \dots, a_m \in [0, 1]$ uniformly at random (independently).
    WLOG, assume $a_1 < a_2 < \dots < a_m$.
  \item Assume we have a hash function $h$ s.t. for each element (key),
    we can map it to $h(k) \in [0, 1]$ uniformly at random.
  \item Define $I(i) = \begin{cases}
      (a_{i-1}, a_i], & i > 1, \\
      (a_m, 1] \cup [0, a_1], & i = 1.
    \end{cases}$ If $h(k) \in I(i)$, then put $k$ into the slot $i$.
    (Note: the complexity of this operation is $\Ord(\log m)$)
  \item To add one more slot:
    \begin{itemize}
      \item Sample $a_{m+1}$ from $[0, 1]$ uniformly at random.
      \item Assume $a_{m+1} \in I(i)$, then use $a_{m+1}$ as a pivot to break
        $I(i)$ into two part, and move all elements whose $h(k)$ is in the
        first part into the new slot.
      \item Since $\Ev{\abs{I(i)}} = \frac{1}{m+1}$, we have
        $\Ev{\text{\#  of elements to move}} = \frac{n}{m+1}$.
    \end{itemize}
\end{enumerate}

\begin{theorem}
  $\Prob*{\max_i (a_{i+1} - a_i) \ge \frac{2 \ln m}{m}}$ is small.
  \begin{proof}
    \begin{align*}
      \Prob*{\max_i (a_{i+1} - a_i) \ge \frac{2 \ln m}{m}}
      &\le \left(1 - \frac{2 \ln m}{m}\right)^{m-2} \\
      &\le \frac{1}{m^2} \qedhere
    \end{align*}
  \end{proof}
\end{theorem}
\begin{theorem}
  With prob. $\Omega(1)$, $\exists$ one interval of size $\le \frac{1}{m^2}$.
\end{theorem}
%%%%%%%%%%%%%%%%%%%%%%%%%%%%%%%%%%%%%%%%%%%%%
% \bibliographystyle{plain}
% \bibliography{journal.bib}
% \begin{thebibliography}{99}
% \bibitem[1]{ex}\url{http://www.example.com/}
% \end{thebibliography}
\end{document}
